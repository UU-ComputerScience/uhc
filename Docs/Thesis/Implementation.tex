\chapter{Implementation}
\label{Implementation}
This chapter details the implementation of the type system HML using \rcore. Due to the amount of code it will be a tutorial style chapter which introduces the new pieces in small increments. The full specification for the algorithms are given for comparisons reason in appendix \ref{appendix:formal}.
\section{Types}
The implementation of the HML types can be easily derived from those in figure \ref{types}. The type scheme $\varphi$ and the SystemF $\sigma$ are extended to support the sugared syntaxes at the end of section \ref{sec:Types}. This is done to simplify some phases of the implementation.

\begin{figure}[H]
\begin{minipage}[t]{0.4\linewidth}
\begin{code}
  con Sugar
     prefix  :: Prefix
     tyExpr  :: TyScheme
\end{code}
\end{minipage}
\begin{minipage}[t]{0.6\linewidth}
\begin{code}
  con Forall
    qu        :: TyQu
    tyVar     :: [HsName]
    tyExpr    :: TyExpr 
\end{code} 
\end{minipage}
\caption{Addition to the $\varphi$ and $\sigma$ types.}
\label{abs:fig:types}
\end{figure}

Along with these, two helper functions \emph{sugar} and \emph{desugar} are defined. The implementation of these two functions is not that interesting, however they do hold the property that \emph{desugar . desugar == id}. These two types facilitate the aggregation of quantified values. For instance $\forall \alpha . \forall \beta. \alpha \rightarrow \beta$ can be represented as $\forall \alpha \beta . \alpha \rightarrow \beta$. Doing this minimizes the amount of code needed in the implementation, because it makes it easier to in one pattern matching get all the bounded variables.
\subsection{Type Expressions}
The full syntax of the datatype $\sigma$ (SystemF type) is:

\begin{figure}[H]
\begin{minipage}[t]{0.4\linewidth}
\begin{code}
data TyQu
  con TyForall
    
data TyExpr
  con Con
    nm         :: HsName
  con App      
    func       :  TyExpr
    arg        :  TyExpr
  con AppTop   
    tyExpr     :  TyExpr
  con Parens   
    tyExpr     :  TyExpr
  con Ann      
    ann        :: TyExprAnn
    tyExpr     :  TyExpr
  con Wild
\end{code}
\end{minipage}
\begin{minipage}[t]{0.6\linewidth}
\begin{code}
  con Mono
  con Var
    nm         :: HsName
  con VarWild  
    mm         :: HsName
  con Quant    
    qu         :: TyQu
    tyVar      :: HsName
    tyExpr     :  TyExpr
  con Forall   
    qu         :: TyQu
    tyVar      :: [HsName]
    tyExpr     :  TyExpr    
  con Row      
    rowTyExpr  :  RowTyExpr
\end{code}
\end{minipage}
\caption{Datatype description of $\sigma$.}
\label{abs:fig:tyexpr}
\end{figure}

Most of these are pretty standard with the exception of \emph{Row} and \emph{AppTop}. The \emph{Row} constructor is used to represent the type of a Tuple, this will be explained later in Chapter \ref{cap:Extensions}. 
The \emph{AppTop} constructor is used purely for pretty printing purposes, in particular when pretty printing Constructor applications. The |(->)| constructor is not treated specially in this version of EH, which means |alpha -> beta| is encoded as |((-> alpha) beta)|. The \emph{App} found under an \emph{AppTop} is pretty printed differently to show the expected type |alpha -> beta|.

\section{Inference}
Like any \rcore program, the first step is to define the datatypes and interfaces needed. Since type inferencing is done on expressions, we first have to introduce the expression datatype:

\begin{figure}[H]
\begin{minipage}[t]{0.4\linewidth}
\begin{code}
data Expr
  con IConst
    int            :: Int
  con CConst       
    char           :: Char
  con Con          
    nm             :: HsName
  con Var          
    nm             :: HsName
  con App          
    func           :  Expr
    arg            :  Expr
  con Let          
    isStrict       :: Bool
    decls          :  Decls
    body           :  Expr
  con Lam          
    arg            :  PatExpr
    body           :  Expr
  con AppTop       
    expr           :  Expr
\end{code}
\end{minipage}
\begin{minipage}[t]{0.6\linewidth}
\begin{code}
  con Parens
    expr           :  Expr
  con TypeAs       
    tyExpr         :: TyExpr
    expr           :  Expr
  con DataFields   
    dataFieldExpr  :: DataFieldExpr

itf Expr
  visit infer
    inh pre  :: Prefix
    inh env  :: Gamma
    inh ast  :: Expr
    inh exp  :: Expr
    inh frs  :: Int
    syn pre  :: Prefix
    syn sub  :: Env
    syn res  :: Maybe TyScheme
    syn frs  :: Int
\end{code}
\end{minipage}
\caption{Datatype and interface description of the expression type.}
\label{abs:fig:expr}
\end{figure}

Although datatypes are defined in figure \ref{abs:fig:expr}, there are no declarations for \emph{case} statements. This was done intentionally as \emph{case} statements are an extension of the base implementation, which will be covered in chapter \ref{cap:Extensions}. This version of EH only has build-in support for the \emph{Int} and \emph{Char} primitive types.

The interface is rather simple, in the \emph{infer} visit the inherited attributes are declared. These consist of a Gamma $\Gamma$, the Prefix $Q$ and expression to infer. The \emph{frs} attribute is used in almost every visit to generate fresh variables. If this value is not passed in a visit then it could trigger a type clash. Generating fresh variables in such a manner has the benefit that it makes every free variable unique, which in turn makes it easier to track down problems. With the exception of type inference for variables, every case in the semantics of $\emph{Expr}$ should return types that are fully quantified. In other words, types which have no free variables. Variables are never directly returned to the user. They are always used as part of a bigger expression. The expression using the variable is responsible to generalizing the type of the variable and thus returning a quantified type.

Before continuing a small explanation of how EH is constructed is needed:
\subsection{Ordering}
Whenever a Haskell file is desugared into EH, some reordering is done. In EH function, data declarations and type synonyms are always displayed before any function or other entities that uses them. Or alternatively, for every entity (where entity can be anything that refers to another type/function) a node is created and a dependency graph is made, which is then \emph{Topologically sorted}, that is, for every pair of edge \emph{uv} (\emph{u} is used in \emph{v}) \emph{u} comes before \emph{v}. It is worth noting that datatype and type synonym declarations are always put at the top of the file, before the function declarations.

Mutual recursion like the ones below are handled a bit differently. But they still need to be supported non the less.

\begin{figure}
\begin{code}
data Foo = Foo Bar
data Bar = Bar Foo | NoBar

foo = bar
bar = foo
\end{code}
\caption{Example of mutual recursion}
\label{fig:mutrecurse}
\end{figure}

The functions |foo| and |bar| never terminate, but they should however type check. For datatypes nothing special needs to be done to make EH support mutual recursion. When processing datatypes, the declarations are parsed and constructor functions created. The desugaring to EH already does the sanity checking of making sure all types exist. The same is done for functions and type synonyms. The EH syntax for \emph{Let} allows the binding of multiple \emph{Decl} at once. For the example this means that there will be one \emph{Let} binding with both \emph{foo} and \emph{bar} at the same time.

Any Haskell module after being desugared to EH is turned into one big \emph{Let} statement, terminating in either 0 or \emph{main} depending on if an actual main function is defined in the module.

Figure \ref{fig:mutrecurse}, which does not define a main is desugered into:

\begin{code}
let main = 0
in let data Bar  = Bar Foo | NoBar
       data Foo  = Foo Bar
   in let foo = bar
          bar = foo
      in 0
\end{code}

The function \emph{main} is the top-level declaration since it is not dependent on anything, in every other cases where main is defined it will be put somewhere inside the dependency graph. In such cases the top level \emph{Let} will terminate \emph{in} main.
\subsection{Semantics}

The ordering that EH brings to a Haskell module is quite handy, especially for \ag code. \ag compilers do computations on attributes defined on the nodes of a tree. A Haskell file desugared into an EH file is now a Top-down tree, where the only case that needs some special handling is mutual recursion. 

This makes certain tasks easier. An example is when looking up a value in the environment if the value cannot be found at the position it is used, then it is really not defined. There is no need to look any further. If a specific element cannot be type checked then it also means there is no reason to try any of its siblings, since they depend on the element that failed type checking.

\subsection{defining the data semantics}
With the explanation of EH the data semantics for inference can be incrementally defined.

\paragraph{header and variables}
The case for variable and constants can be easily defined as:
\begin{code}
datasem Expr monad IO
    default? sub = const []
    default? pre = last
    default? frs = last
    default? res = const Nothing
    clause Var
        lhs.res =
          case lookup loc.nm (unGam lhs.env) of
           Just a  -> return a
           Nothing -> error ("Variable '" plus show loc.nm plus "' not found")
    clause IConst
       lhs.res = return $ TyScheme_SystemF $ TyExpr_Con $ mkName "Int"
    clause CConst
       lhs.res = return $ TyScheme_SystemF $ TyExpr_Con $ mkName "Char"
\end{code}

The definitions of the default rules are simply:
\begin{description}
\item[\textbf{sub}] If no substitutions are defined, then return an empty list
\item[\textbf{pre}] If the prefix is not changed, then return the last value seen
\item[\textbf{frs}] If no new variables were introduced, then return the last value (which should also be the highest value)
\item[\textbf{res}] If no type is inferred, return Nothing
\end{description}

When inferring the type of a variable, the only thing that needs to be done is looking up the variable in the $\Gamma$. If the value is not found then the variable is undefined and an error is returned. Error reporting is admittedly crude in the implementation, but it gives adequate feedback.

For the Integer (\emph{IConst}) clause the \emph{Int} type is returned and for the Char (\emph{CConst}) clause the type \emph{Char}.

\paragraph{Constructors}
For any other constructor the name of the constructor is looked up. But as mentioned in section \ref{renaming} the types of constructors are $\alpha-$renamed.

\begin{code}
clause Con
   loc.res = maybe (error ...) id (lookup loc.nm (unGam lhs.env))
   (loc.res2, lhs.frs, _) = renameBound lhs.frs loc.res
   lhs.res = return loc.res2
\end{code}

\paragraph{Applications}
Typing applications is the first part where the code gets a bit complex:

\begin{code}
clause App
   loc.frs              = lhs.frs
   (loc.a1,  loc.frs1)  = fresh loc.frs
   (loc.a2,  loc.frs2)  = fresh loc.frs1
   (loc.b ,  loc.frs3)  = fresh loc.frs2
\end{code}
Three fresh variabels $\alpha_1, \alpha_2$ and $\beta$ are first declared.

\begin{code}
func.frs = loc.frs3
func.pre = lhs.pre
func.env = lhs.env
    
match True    = isJust func.res       
loc.func_res  = fromJust func.res
\end{code}
The type of the \emph{function} then is inferred. If the inference is successful  it then proceeds to the next step.

\begin{code}       
arg.frs = func.frs
arg.pre = func.pre
    
arg.env = appAll func.sub lhs.env
 
match True   = isJust arg.res   
loc.arg_res  = fromJust arg.res
\end{code}
The environment $\Gamma$ is then updated by applying the substitutions returned as the result of inferring the type of \emph{func} to the initial environment. This is an important step, it updates local bindings within the environment with information which might have been discovered when inferring the type of the function. Variables can be shared, for instance \emph{f} is shared between both the \emph{function} and the \emph{argument} in the application |\f -> (\x -> f x) (\x -> f x)|. After this the type of the argument can be inferred.
       
\begin{code}
loc.temp1 = extend (arg.pre
                   ,Scheme_Simple loc.a1 (appAll arg.sub loc.func_res))
loc.temp2 = extend (fst loc.temp1
                   ,Scheme_Simple loc.a2 (loc.arg_res))
loc.temp3 = extend (fst loc.temp2
                   , Scheme_Simple loc.b  (TyScheme_Bottom))
    
(loc.q2', loc.e2') 
   = (fst loc.temp3
     ,(snd loc.temp1) `munion` (snd loc.temp2) `munion` (snd loc.temp3) `munion` func.sub)
\end{code}
The type of the function is updated by applying the substitution from the inference phase of the argument to it. This is because the inference of the argument might have given more type information about the function as well. The environment is then extended by adding three new bindings \ba{(\beta \geq \bot), (\alpha_2 \geq arg.res), (\alpha_1 \geq (appAll arg.sub func.res))}

\begin{code}
child u : Unify = unify
u.pre   = loc.q2'
       
u.exp1  = appAll loc.e2' (TyExpr_Var loc.a1)
u.exp2  =  let begin = appAll loc.e2' (TyExpr_Var loc.a2)
           in begin `mkArrow` (TyExpr_Var loc.b)
u.frs   = arg.frs
       
loc.q3  = u.pre
loc.e3  = u.sub
\end{code}
$\alpha_1$ and $\alpha_2$ represent the function and argument types respectively. $\beta$ is the return type which is to be discovered. To find out what $\beta$ is, the function type $\alpha_1$ is unified with the type $\alpha_2 \rightarrow \beta$, which is the shape that $\alpha_1$ needs to have. If unification is successful then $\beta$ will contain the result of the application $\alpha_1 \hspace{5pt} \alpha_2$.


\begin{code}       
(loc.q4, loc.q5) = curry split loc.q3 (domain lhs.pre)
 
lhs.res =  let  b    = TyExpr_Var loc.b
                e    = appAll (loc.e3 `munion` loc.e2') b
                ret  = TyScheme_Sugar loc.q5 (TyScheme_SystemF e)
           in return (desugar ret)
loc.temp_sub =  let subs = func.sub `munion` arg.sub 
                in subs `munion` loc.e3
lhs.sub = loc.temp_sub
lhs.pre = loc.q4
lhs.frs = u.frs
\end{code}
Should unification succeed then $\beta$ is the result type. Before it can be returned it needs to be fully quantified. This is done by generalizing based on the initial prefix. Generalization is done using the \emph{split} function. The final prefix is split in two based on the domain of the initial prefix. The first prefix contains the already existing prefixes that were updated, whereas the second prefix would contain only new entries (prefixes for local types introduced in the application).

\paragraph{Let binding}
Type checking a Let binding is very straight forward because of how EH is structured.

\begin{code}
(loc.frs, loc.binds)  =  let  nms          = getBindLHs loc.decls
                              (vars, frs)  = freshM lhs.frs (length nms)
                         in (frs, zipWith (\a b->(a, mkQVar b)) nms vars)
loc.env   =  let binds = map (uncurry minsert) loc.binds 
             in Gamma (foldr (.) id binds (unGam lhs.env))
\end{code}
The bindings are preprocessed in order to support mutual recursive functions. When functions are dependent on each other they are placed in the same Let. They cannot just be sequentially typed without a temporary value being present in the environment for the all sibling functions. This is why a new polymorphic type is added to the $\Gamma$ for every binding. The function \emph{freshM} generates a list of fresh variables and the function \emph{mkQVar} generates quantified variables.

\begin{code}
decls.pre  = lhs.pre
decls.env  = loc.env
decls.frs  = loc.frs
decls.sig  = Gamma []
\end{code}
After this preprocessing step all the declarations can now be handled sequentially with no problem.

\begin{code}
body.pre  = decls.pre
body.env  = decls.env 
body.frs  = decls.frs

lhs.pre   = body.pre
lhs.res   = body.res
lhs.sub   = decls.sub `munion` body.sub
lhs.frs   = body.frs
\end{code}
After every declaration in the let is processed then the body can be typechecked.

\paragraph{Lambda abstraction}
Type checking a $\lambda-abstraction$ requires that is possible to type check pattern expressions. However for brevity's sake pattern expression inference will not be covered in-depth. Essentially inference for pattern expressions work the same way as for normal expressions, since patterns are a subset of the constructs of the expression data type. There is one major and important difference however: Types in pattern expressions are always quantified in the prefix environment/attribute whereas types in the expression inference are always fully quantified in the type itself. The reason for this is that the results of a pattern expressions are never directly used as a result, and as a result it is much easier to use them when they are quantified in the prefix. Variables can also be shared, if they were to be returned fully quantified then all of the variables would be distinct instead of being them same.

\begin{code}
clause Lam
    arg.env    = lhs.env
    arg.pre    = lhs.pre
    arg.frs    = lhs.frs
   
    loc.exp_a  = ftype arg.res
    (loc.b, loc.frs2) = fresh arg.frs
\end{code}
The type of the argument to the lambda has to be inferred before anything else. This argument can contain patterns, which are returned inside the $\Gamma$ environment of the pattern expression. Pattern expressions can introduce new variables into scope, which is why pattern expressions have the only interface that also return a $\Gamma$. Lastly a fresh variable $\beta$ is also generated.
       
\begin{code}
loc.env'  = appAll arg.sub arg.env
          
body.frs  = loc.frs2
body.env  = loc.env'
body.pre  = arg.pre
    
loc.env_full = arg.sub `munion` body.sub
\end{code}
After inferring the type of the argument and updating the environment we can then infer the type of the body. The substitution environments of the two are then added together.     

\begin{code}
loc.tau =  let x = appAll loc.env_full loc.exp_a
           in if  isUnQualTy x || not (isVar loc.arg)
                  then x
                  else error "The resulting type is not a Tau type"
                     
loc.phi1 = case (loc.tau `seq` body.res) of
             Nothing  -> error "Inference of Expressions failed in Lam"
             Just x   -> x
\end{code}
After inferring the type of the body some extra information is gained on the argument of the abstraction. The next step is to check if the argument when applied to the full substitution environment is still of suitable type. In this case suitable is defined as being an unquantified type or the original must hav been a pattern expression.

\begin{code}
(loc.pre, loc.phi1')  = deep_explode body.pre loc.phi1
(loc.q2,  loc.q3)     =  let range = domain lhs.pre `reachedBy` loc.env_full
                         in curry split loc.pre range
(loc.q3', loc.e3')    = extend (loc.q3, (Scheme_Simple loc.b loc.phi1'))
    
lhs.res =  let  b'     = appAll loc.e3' (TyExpr_Var loc.b)
                a'     = appAll body.sub loc.exp_a
                arrow  = a' `mkArrow` b'
           in desugar (TyScheme_Sugar loc.q3' (TyScheme_SystemF arrow))
lhs.sub  = body.sub
lhs.pre  = loc.q2
lhs.frs  = body.frs
\end{code}
The prefix is then separated from the rest of the type inferred for the body of the expression. |deep_explode| is a more aggressive form of the |explode| function, in that it removes the quantifications always, and not just when the normal form of the type is a |bot|. The original specification does not have this step and because of this produces very complex type schemes. Adding this step does not reduce the flexibility of the type schemes but does increase the readability of the result type. 
During generalization some extra work is needed to assure than when a type is present in both the result of a call to inference and the resulting prefix, that it is qualified in the prefix instead of the type.
The |reachedBy| function just returns the first input argument containing the |dom Q| and the |codomain| of every substitution whose domain intersects with the domain of |Q| or the |codomain| with the |codomain| of another substitution which does. This is important for expressions such as |\f -> (\x -> f x) (\x -> f x)|. The original specification does not have this step, and because of this incorrectly type checks this function. The original specification is a little too eager when it comes to generalizations. After this the final type for the abstraction can be constructed.

\section{Unification}
Unification is the core part of this thesis. Unification is the first routine which requires traversing of two trees at the same time. The interface for unification is defined as:

\begin{code}
itf Unify
  visit unify
    inh pre   :: Prefix
    inh exp1  :: TyExpr
    inh exp2  :: TyExpr
    inh frs   :: Int
    syn pre   :: Prefix
    syn sub   :: Env
    syn frs   :: Int
\end{code}

The goal of the visit is to unify the type |exp1| with the type |exp2| under the prefix \emph{pre}. The result of a unification, if successful, is a substitution environment \emph{sub} and the prefix under which the two type are unified.

\begin{code}
unify = sem unify : Unify monad IO    
          visit unify
             default? pre  = last
             default? frs  = last
             default? sub  = const (error "Generic Unify failure")
             clause Main_unify
               loc.nfTy1  = isNf lhs.exp1
               loc.nfTy2  = isNf lhs.exp2
               loc.nf     = loc.nfTy1 && loc.nfTy2
               internal nf_check
                 clause s_unify
                   match True  = loc.nf
\end{code}
Before types can be unified there is a check to see if both types are in Normal form. Even though all SystemF types have been promoted to flexible types, part of the systemF types are still embedded inside the flexible type(e.g. applications, constructors etc). This is why the inherited attributes are all still SystemF types. There is instead of providing an empty list as the default result of substitutions an error message with a fairly generic error. If the implementation is correct then this should never be printed.

\begin{code}
internal ty_check
  clause LeftTop
    match TyExpr.AppTop@app = lhs.exp1
    
    child u : Unify = unify
    
    u.pre    = lhs.pre
    u.frs    = lhs.frs
    u.exp1   = app.tyExpr
    u.exp2   = lhs.exp2
             
    lhs.frs  = u.frs
    lhs.pre  = u.pre
    lhs.sub  = u.sub
\end{code}
The clause \emph{LeftTop} and \emph{RightTop} are there just to strip away the \emph{AppTop} constructor from types. The \emph{AppTop} constructor is only used for pretty printing. After stripping the constructor the new expression is used to further unification. Though only \emph{LeftTop} is given \emph{RightTop} is the same, only ranging over the |lhs.exp2| attribute.

\begin{code}
clause LeftParens
  match TyExpr.Parens@app = lhs.exp1
  
  child u : Unify = unify
  
  u.pre    = lhs.pre
  u.frs    = lhs.frs
  u.exp1   = app.tyExpr
  u.exp2   = lhs.exp2
           
  lhs.frs  = u.frs
  lhs.pre  = u.pre
  lhs.sub  = u.sub
\end{code}
The actual physical parenthesis are also not needed. After parsing the order of the type in the AST already denotes the intended meaning of the parentheses. They are retained for pretty printing but are otherwise unneeded. Clause \emph{LeftParens} and \emph{RightParens} are there to strip them.

\begin{code}
clause vars
  match TyExpr.Var@var1 = lhs.exp1
  match TyExpr.Var@var2 = lhs.exp2
  
  internal case_check
    clause eq_vars
      match True = var1.nm == var2.nm
      lhs.sub = []
\end{code}
When both expressions are variables there are a number of choices. The first of which is when the variables in both expression are the same. In that case unification for them succeeds due to the fact that the types are already equal. A empty substitution is therefore returned.

\begin{code}
clause rest_vars
  loc.q  = lhs.pre
  loc.p1 = loc.q `contains` var1.nm
  loc.p2 = loc.q `contains` var2.nm
  internal var_check
    loc.isJust    = isJust loc.p1 && isJust loc.p2
    clause pure_var
      match False = loc.isJust
      lhs.sub = error ... -- Binding not found
\end{code}
When the variables are not the same, then a check to see if both variables are bound by the environment(the Prefix) is performed. If they are not in the prefix then unification cannot proceed because information on the bounds of the variables are missing. This case is handled by generating an error and printing the variables and the prefix in question.

\begin{code} 
clause ty_var
  match True  = loc.isJust
  loc.phi1 = fromJust loc.p1
  loc.phi2 = fromJust loc.p2
  loc.exists =  let  v1  = occursCheck var1.nm loc.q (toScheme loc.phi2)
                     v2  = occursCheck var2.nm loc.q (toScheme loc.phi1)
                in v1 || v2
  internal dom_check
    clause s_dom_check
      match True  = loc.exists
      lhs.sub     = error "Domain check failed in Unify"
\end{code}
In case both of the variables are bound in the prefix, their bounds are extracted and then bounded to the local attributes \emph{|phi1|} and \emph{|phi2|}. The next step is to check that neither bounds contain any reference to the other variable, in other words: |var1 _notin domain(loc.phi1) \cap var2 _notin domain(loc.phi2)|. If this assertion is untrue then an error message is generated. This process is called rather aptly the "occurs check".
This "occurs check" is usually written as |(dom (Q/phi))|. It is meant to return the \emph{useful} domain of |Q| with respect to |phi|\cite{HML}. As an example |(dom ((gamma >= bot, beta >= forall delta . delta -> gamma)/(forall alpha . alpha -> beta)) )| is |{beta, gamma}| even though |gamma _notin ftv(forall alpha . alpha -> beta)|. 

\begin{code}
clause f_dom_check
  match False = loc.exists
  
  child s : UnifyScheme = unifyScheme
  
  s.pre    = loc.q
  s.frs    = lhs.frs
  s.ty1    = dropQuant loc.phi1
  s.ty2    = dropQuant loc.phi2
           
  loc.q1   = s.pre
  loc.e1   = s.sub
  loc.phi  = s.ty
\end{code}
After extracting both type schemes a call to UnifyScheme is made in order to get the resulting unified type of those two type schemes. \emph{dropQuant} in the call above is actually defined as the identity function. Its purpose is to illustrate that there is a conversion or coercion from a quantified type scheme $\hat{\varphi}$ to a normal type scheme $\varphi$ going on. Because a quantified type scheme is just a type synonym in this implementation the function is defined as |id|. Finally the result of the child \emph{s} is bound to more sensible names.

\begin{code}                            
(loc.q2, loc.e2)
  = update (loc.q1
           ,Right (var1.nm, TyScheme_SystemF (TyExpr_Var var2.nm)))
(loc.q3, loc.e3) 
  = update (loc.q2, Left (Scheme_Simple var2.nm loc.phi)) 

lhs.frs = s.frs
lhs.pre = loc.q3
lhs.sub = loc.e3 `munion'` loc.e2 `munion'` loc.e1
\end{code}
The new prefix is updated with the new information. It starts by updating the prefix returned from the call to UnifyScheme with the substitution \ba{|var1 := var2|}. This is done because the two variables are now equal, so the reference to one of them can be removed. Since we are trying to find out what \emph{|var1|} is, we can safely write that \emph{|var1|} is now \emph{|var2|}. \emph{|var2|} is then updated with the new bound |(var2 >= loc.phi)|. This updates the value of \emph{|var2|} to the result of the Unification of the two type schemes. Lastly all the substitutions gathered are combined with the operator |`munion'`| which inserts the substitutions in a place where the invariant is still maintained. After the substitutions are combined they are then returned.

\begin{code}
clause cons
  match TyExpr.Con@var1 = lhs.exp1
  match TyExpr.Con@var2 = lhs.exp2
  
  loc.eq = var1.nm == var2.nm
  
  internal con_check
     clause succeeded1
       match True  = loc.eq
       lhs.sub     = []
     clause failed1
       match False = loc.eq
       lhs.sub = error -- type mismatch error
\end{code}
When both expressions are constructors, then a check is done to see if the name of the constructors match. If they do not match then an error is thrown, however if they match then return with an empty substitution.

\begin{code}
clause apptype
  match TyExpr.App@app1 = lhs.exp1
  match TyExpr.App@app2 = lhs.exp2
  
  child uf : Unify = unify
  
  uf.frs   = lhs.frs
  uf.pre   = lhs.pre
  uf.exp1  = app1.func
  uf.exp2  = app2.func
  
  child ua : Unify = unify
  
  ua.frs   = uf.frs
  ua.pre   = uf.pre
  ua.exp1  = appAll uf.sub app1.arg
  ua.exp2  = appAll uf.sub app2.arg
           
  lhs.pre  = ua.pre
  lhs.sub  = uf.sub `munion` ua.sub
\end{code}
Applications are also easy: If both types are an Application then two calls to Unify needs to be made. For example when unifying the expressions $f_1 \hspace{5pt} e_1$ and $f_2 \hspace{5pt} e_2$, then first $f_1$ and $f_2$ have to be unified. In case this succeeds then the substitutions found by this unification is applied to the types $e_1$ and $e_2$ to propagate any new information found. The resulting types are then unified together to produce the final result.

\begin{code}
clause vartype   
  internal varchoice
    clause leftvar
      match TyExpr.Var@var1 = lhs.exp1
      loc.expr  = lhs.exp2
      loc.varnm = var1.nm

      loc.q   = lhs.pre
      loc.phi = loc.q `contains` loc.varnm
      loc.occurs = occursCheck  loc.varnm loc.q 
                                (TyScheme_SystemF loc.expr)
\end{code}
This next case deals with when only one of the expressions is a variable and the other an arbitrary expression (but not a variable). What to do when the first expression |(lhs.exp1)| is a variable is handled here, but there is a similar clause for when the second expression |(lhs.exp2)| is a variable. 
First the binding to the variable is looked up in the prefix. Afterwards there is a check to see if the domain of that type contains any reference to the variable itself (infinite type check).
                           
\begin{code}
internal occurs_check
  clause valid
    match False = loc.occurs
    loc.phi' = fromJust loc.phi
    
    loc.p = loc.expr
    
    child s : Subsume = subsume
    
    s.pre  = loc.q
    s.exp  = loc.p
    s.frs  = lhs.frs
    s.ty   = dropQuant loc.phi'
    
    (loc.q2, loc.e2) =  let  a    = appAll s.sub loc.expr
                             sub  =  (loc.varnm
                                     ,TyScheme_SystemF (mkTop a))
                        in update (s.pre, Right sub)
\end{code}
In case the occurs check succeeds, try to instantiate the SystemF expression with the type scheme in the bounds. This is the same clause that handles the type checking of higher-rank types. If instantiation succeeds then a list of substitutions indicating how to instantiate the SystemF expression (\emph{loc.expr}) is returned.
The value of the variable that is being unified with the expression is updated using the substitutions, replacing that variable with the instantiated expression.

\begin{code}
  lhs.pre  = loc.q2
  loc.ret  = loc.e2 `munion` s.sub
  lhs.sub  = loc.ret
  lhs.frs  = s.frs  
clause invalid
  match True  = loc.occurs
  lhs.sub     = error ... -- infinite type error
\end{code}
The updated prefix and complete substitution list is then returned. In case the occurs check fails, then an error message indicating there was a problem with an infinite type is generated.

\begin{code}
clause quant
  match TyExpr.Quant@var1 = lhs.exp1
  match TyExpr.Quant@var2 = lhs.exp2
  
  child u : Unify = unify
  
  (loc.c, loc.frs1) = fresh lhs.frs
  loc.varc = mkSkolem loc.c
  u.frs   = loc.frs1
  u.pre   = lhs.pre
  u.exp1  = app (var1.tyVar, loc.varc) var1.tyExpr
  u.exp2  = app (var2.tyVar, loc.varc) var2.tyExpr
          
  loc.q1  = u.pre
  loc.e1  = u.sub
\end{code}
If the goal was to unify two quantified SystemF expressions (should not occur naturally in the call to unify since quantified systemF expressions are promoted to type schemes, but can occur when unifying function annotations with the inferred type.) the quantifier of both types are replaced with the same skolem variable. The idea is to check if both types have the same structure. Since they are equally quantified, if they have the same structure than they are the same type.
A skolem variable is a type that only unifies with itself and other normal variables. It is sometimes also referred to as a \emph{Rigid} type\footnote{Those familiar with GHC will recognize this. Although GHC $\geq$ 7.0 also refer to them as skolem variables}. In this implementation a special constructor (names starting with a lowercase "s") is used to implement skolem variables.
\begin{code}                       
loc.check = loc.c `elem` (codomain loc.e1 ++ codomain loc.q1)
internal skol_check
   clause succeeded
     match False  = loc.check
     lhs.pre  = loc.q1
     lhs.sub  = loc.e1
     lhs.frs  = u.frs
   clause failed
     match (loc.bla, True) = ((), loc.check)
     lhs.sub = const (error ...) loc.bla -- skolem leak error
\end{code}
At the end a check is performed to see if the skolem variable used has leaked out of the environment by checking the codomain of the returned substitution and prefix.

\begin{code}
    clause mismatch
      lhs.sub = error ... -- type mismatch error
clause f_unify
  match (loc.bla, False) = ((), loc.nf)
  lhs.sub = const (error ...) loc.bla -- not in normal form
\end{code}

\section{Instantiation}
Another major part of the type system is instantiation. Instantiation (referred to as subsume) tries to find if the type scheme \emph{lhs.ty} can be modified into the SystemF type \emph{lhs.exp}. Instantiation is critical to the support of higher-rank types.

The interface to subsume is:
\begin{code}
itf Subsume
  visit subsume
    inh pre  :: Prefix
    inh exp  :: TyExpr
    inh ty   :: TyScheme
    inh frs  :: Int
    syn pre  :: Prefix
    syn sub  :: Env
    syn frs  :: Int
\end{code}
The only important inputs are the Prefix, TyExpr and TyScheme to use during instantiation. The result is a list of substitutions (Env) which when applied to the type scheme (ty) would instantiate the type to SystemF type (exp) under the  prefix (pre).


\begin{code}
subsume = sem subsume : Subsume monad IO 
             visit subsume
               default? pre  =  last
               default? frs  =  last
               clause Main_subsume
                loc.ty    =  nf lhs.ty
                loc.nfTy  =  isNf lhs.exp
                loc.nfTs  =  isNf loc.ty
                loc.nf    =  loc.nfTy && loc.nfTs
                internal nf_check
                    clause s_subsume
                      match True                   =  loc.nf
                      match TyExpr.Forall@exp      =  mkForall lhs.exp
                      match TyScheme.Sugar@scheme  =  sugar loc.ty
\end{code}
In order for instantiation to be performed, both type have to be in normal form. If this is not the case, they are converted into normal form. Because of the promotion of SystemF types, sometimes types that are valid but not in normal form are passed to this function. This conversion reduces the amount of steps needed to instantiate the type. This version of subsume also handles types not in normal form correctly, but the checks serve as a sanity check for the rest of the code, to make sure no type is  generated that is too complicated or incorrectly nested. For example a type that violates the semantics of type schemes would be instantiated incorrectly if this check is not present, but a type which is not in normal form but does not violate the semantics would be instantiated correctly.
The calls to \emph{mkForall} and \emph{sugar} allows the binding of every \emph{outer-quantifiers} in one call instead of one by one removing them recursively in AG code.

\begin{code}
loc.q   = lhs.pre
loc.q2  = scheme.prefix

-- generate fresh skolem variables
(loc.c, loc.frs1) = freshS lhs.frs (length exp.tyVar)
loc.qq2 = loc.q ++ loc.q2

loc.p1  = dropScheme scheme.tyExpr
loc.p2  =  let skol = (zipWith (\a b->(a, mkSkolem b)) exp.tyVar loc.c)
           in  appAll skol exp.tyExpr
\end{code}
Since the goal is to instantiate the type scheme (\emph{scheme}) to the SystemF type (\emph{exp}), the type scheme bounds of that type are added to the prefix resulting in the value (|loc.qq2|). 
The binding of attribute \emph{|p1|} has a call to \emph{dropScheme} which does not do anything but ensure that the type is a SystemF type wrapped inside a type scheme (e.g. that the scheme.tyExpr type is not $\bot$ or any other TyScheme value). \emph{|p2|} has an interesting construct but it essentially does skolemization. It replaces every bound type variables in the type to a skolem variable.

\begin{code}
child u : Unify = unify
u.pre  = loc.qq2

u.exp1  = loc.p1
u.exp2  = loc.p2
u.frs   = loc.frs1
        
loc.q1  = u.pre
loc.e1  = u.sub
\end{code}
After preparing the two new input functions, the new unify co-routine is declared and invoked. The original specifications for |subsume| incorrectly switches the order of |p1| and |p2|. This is a critical error in that it switches the type that is to be instantiated with the type it should be instantiated with. The effect of this is that |generalization| generates incorrect types or incorrectly quantified types such as |(alpha >= bot). Bool|.
This ultimately results in information loss and an error depending on the complexity of the expression.

\begin{code}   
(loc.q3, loc.q4) = split (loc.q1, domain loc.q)
loc.e2 =  let x = domain loc.q4
          in loc.e1 `remove` x
loc.codom = let except = nub (domain loc.q \\ prefixDom loc.q)
                skols  = filter ((`elem` except).fst) loc.e1
            in  (nub (skolems loc.e2 ++ skolems loc.q3)) \\ skolems skols
loc.check  = any (elems loc.codom) loc.c
\end{code}
The resulting prefix \emph{|loc.q1|} is generalized based on the original prefix. \emph{|loc.q3|} contains the updated values (if any) of the prefix that was originally given to the visit, whereas \emph{|loc.q4|} contains any prefixes which do not have any direct effect on the instantiation. Using this prefix we can then apply scoping to the list of substitutions received from the unification call. This way the only substitutions returned are those that have a direct effect on the instantiation of the input type.
The local attribute \emph{codom} contains every skolem variable in the substitution environment and the prefix returned from unify. The broader definition of the domain of prefixes cause a problem here. This broader definition mistakenly exports substitutions that shouldn't be exported. This is corrected by using the |prefixDom| function, which defines the domain of a prefix to be just the |variables| to the left size of the first |>=|, |dom (alpha >= (beta >= bot)) = {alpha}|. With this definition a list of |exceptions| are generated. Which are a list of variables that should not be checked for a leak. Even if they do contain a skolem variable, they are never used beyond this point. And if they do not contain a skolem variable they most likely will be used. Which is why we cannot simply remove them from |loc.e2|.
The \emph{check} attributes contains the result of the check to see if any of these skolem variables are variables that were defined by the current instantiation call. These two attributes are there to prevent any skolem variables from leaking from instantiation.

\begin{code}
internal skol_check
  clause succeeded
    match False  = loc.check
    lhs.pre  = loc.q3
    lhs.sub  = loc.e2
    lhs.frs  = u.frs
  clause failed
    match (loc.bla, True) = ((), loc.check)
    lhs.sub = const (error ...) loc.bla -- skolem leak error
\end{code}
If no skolems are leaked then just return the prefix and substitution, otherwise give an appropriate error message. Statements are reordered based on dependencies, the local attribute \emph{bla} is defined in the match to insure that the match executes before the error message.

\begin{code}
clause wild_subsume
  match True                        = loc.nf
  match (loc.fo, TyScheme.Bottom@l) = ((), sugar loc.ty)
  lhs.sub = empty
  
clause f_subsume
  match (loc.bla, False) = ((), loc.nf)
  lhs.sub = const (error ...) loc.bla -- skolem form error
\end{code}
The last two clauses are for when the type is instantiated with a $\bot$ and for when the type was not in normal form. In the former we just immediately return with an empty substitution list (nothing needs to be done, the types are already "equal") and in the latter an error is given. The specification of \cite{HML} does not have the $\bot$ case. This results in the inability to type check expressions such as |id id| where the type of id is a type scheme |(alpha >= \bot). alpha -> alpha|. This is believed to be an oversight on the part of the author.

\section{Unify Scheme} 
On occasion two type schemes need to be unified with each other, the interface to do this is defined as:

\begin{code}
itf UnifyScheme
  visit unifyScheme
    inh pre  :: Prefix
    inh ty1  :: TyScheme
    inh ty2  :: TyScheme
    inh frs  :: Int
    syn pre  :: Prefix
    syn sub  :: Env
    syn ty   :: TyScheme
    syn frs  :: Int 
\end{code}

There are two TySchemes as input and the unified TyScheme is returned. The implementation is simple:

\begin{code}
unifyScheme = sem unifyScheme : UnifyScheme monad IO
                visit unifyScheme
                  default? pre = last
                  default? frs = last
                  default? sub = const empty
                  clause Main_unifyScheme
                      loc.b1  = isBottom lhs.ty1
                      loc.b2  = isBottom lhs.ty2
                      internal bot_check
                        clause first_bottom
                            match True  = loc.b1
                            lhs.ty      = lhs.ty2
                        clause second_bottom
                            match True  = loc.b2
                            lhs.ty      = lhs.ty1
\end{code}
When unifying two schemes, if either of them is a $\bot$ then an empty substitution can immediately be returned since anything unifies with $\bot$. 

\begin{code}
clause norm
    match False = loc.b1 || loc.b2
    match TyScheme.Sugar@exp1 = sugar lhs.ty1
    match TyScheme.Sugar@exp2 = sugar lhs.ty2
        
    child u : Unify = unify
        
    u.pre   = lhs.pre ++ exp1.prefix ++ exp2.prefix
    u.exp1  = dropScheme exp1.tyExpr
    u.exp2  = dropScheme exp2.tyExpr
    u.frs   = lhs.frs
            
    loc.q3  = u.pre
    loc.e3  = u.sub
\end{code}
If both type schemes are not a $\bot$ then the prefixes of both type schemes are removed and combined with the input prefix and passed along to a call to Unify.

\begin{code}    
(loc.q4, loc.q5) = split (loc.q3, domain lhs.pre)
    
lhs.frs  = u.frs
lhs.pre  = loc.q4
lhs.sub  = loc.e3
lhs.ty   =  let  ty   = appAll loc.e3 exp1.tyExpr
                 ret  = TyScheme_Sugar loc.q5 ty
            in desugar ret
\end{code}
Generalization can be done after the call to unify by splitting the prefix returned from unify (\emph{u}). Besides returning the substitutions needed to unify the two type scheme, the result of the unification of the two type schemes is also returned. This is done by first applying the substitution to type we were unifying with and then fully quantify it by adding the prefix \emph{|loc.q5|} to the result. Note that $ftv(lhs.ty)=\emptyset$.
