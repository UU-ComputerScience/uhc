\section{Introduction}

%\frame
%{
%  \frametitle{Features of the Beamer Class}
%
%  \begin{itemize}
%  \item<1-> Normal LaTeX class.
%  \item<2-> Easy overlays.
%  \item<3-> No external programs needed.      
%  \end{itemize}
%}

% \begin{frame}[fragile]
% 	\frametitle{By example}
% 
% 	\begin{block}<+->{Consider the recursive function last:}
% 		\begin{lstlisting}
% 		last :: [a] -> a
% 		last []     = error "Last"
% 		last (x:[]) = x
% 		last (x:xs) = last xs
% 		\end{lstlisting}
% 	\end{block}
% 	
% 	Runtime two checks for the Cons (:) constructor are done.
% \end{frame}

\subsection{Type Systems}




\frame{
	\frametitle{The problem}
   
	Type systems are specified using typing rules
   \only<2->{
		\begin{prooftree}
			\AxiomC{$x : \sigma \in \Gamma$}
		\LeftLabel{Var:\quad}
			\UnaryInfC{$\Gamma\vdash x : \sigma$}
		\end{prooftree}
	}
   \only<2->{However..}
}

\frame{
	\frametitle{The problem}
   
	Disconnect between typing rules and implementation
   \begin{itemize}
	\item<1-> Nondeterministic
	\item<2-> Implicit assumptions
	\item<3-> Implementation does not resemble typing rules
	\item<4-> Complexity explodes with size of AST
	\item<5-> A lot of it due to language used
	\end{itemize}
}

\frame{
	\frametitle{Goal}
    
	Implement type system using attribute grammars
	\begin{itemize}
	\item<1-> Easier to understand
	\item<2-> Easier to prove correct
	\item<3-> Easier to document and scale
	\end{itemize}
}

\frame{
	\frametitle{Contributions}

	\begin{itemize}
	\item<1-> Implementation using attribute grammars
	\item<2-> Implementation \& specification for the HML type system for EH
	\end{itemize}
}

\section{Attribute Grammars}

\frame{
	\frametitle{Context Free Grammars}

	\begin{itemize}
     \item<1-> Can only describe syntax
     \item<2-> Cannot specify any context-sensitive conditions
     \item<2-> $a^nb^nc^n$
     \item<3-> A way to define semantics/meaning
	\end{itemize}
}

\frame{
	\frametitle{Context Free Grammars}

	\begin{figure}[H]
\begin{grammar}
      [(colon){$\rightarrow$}]
      [(semicolon)$|$]
      [(comma){}]
      [(period){\\}]
      [(quote){\begin{bf}}{\end{bf}}]
      [(nonterminal){$\langle$}{$\rangle$}]
<Expr>   \hspace{52pt} : <number>; <Expr>, <operator>, <Expr>.
<number> \hspace{42pt} : <digit>;<digit>,<number>.
<digit> \hspace{55pt} : "0";"1";"2";"3";"4";"5";"6";"7";"8";"9".
<relational operator>: "$-$";"$+$".
\end{grammar}
\caption{BNF definition for Expressions}
\label{grammar:bnf:expr}
\end{figure}
}

\frame{
	\frametitle{Parse Trees}

	\begin{figure}[H]
	\centering
	\begin{tikzpicture}[>=stealth']
	\node (r0) [draw, circle] {+};
	\node (r1) [below left=1cm of r0, draw, circle] {2}
	  edge[<-] (r0.south);
	\node (r2) [below right=1cm of r0, draw, circle] {+}
	  edge[<-] (r0.south);
	\node (s0) [below left=1.4cm of r2, draw, circle] {-}
	  edge[<-] (r2.south);
	\node (s1) [below left=1cm of s0, draw, circle] {3}
	  edge[<-] (s0.south);
	\node (s2) [below right=1cm of s0, draw, circle] {5}
	  edge[<-] (s0.south);
	\node (f0) [below right=1.5cm of r2, draw, circle] {-}
	  edge[<-] (r2.south);
	\node (f1) [below left=1cm of f0, draw, circle] {6}
	  edge[<-] (f0.south);
	\node (f2) [below right=1cm of f0, draw, circle] {1}
	  edge[<-] (f0.south);
	\end{tikzpicture}
	\caption{Parse tree example for "2 + (3 - 5) + (6 - 1)"}
	\label{fig.example1.parsetree}
	\end{figure}

	\begin{itemize}
	\item<1-> AG's are additions to CFG
	\item<2-> Nodes are a production or a non-terminal
	\end{itemize}
}

\frame{
	\frametitle{}

}

\section{Result}

\section{Problems and Future work}

\subsection{HML}

\subsection{Ruler-Core}

\section{Conclusion}
