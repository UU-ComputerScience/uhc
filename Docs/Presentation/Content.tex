\section{Introduction}

%\frame
%{
%  \frametitle{Features of the Beamer Class}
%
%  \begin{itemize}
%  \item<1-> Normal LaTeX class.
%  \item<2-> Easy overlays.
%  \item<3-> No external programs needed.      
%  \end{itemize}
%}

% \begin{frame}[fragile]
% 	\frametitle{By example}
% 
% 	\begin{block}<+->{Consider the recursive function last:}
% 		\begin{lstlisting}
% 		last :: [a] -> a
% 		last []     = error "Last"
% 		last (x:[]) = x
% 		last (x:xs) = last xs
% 		\end{lstlisting}
% 	\end{block}
% 	
% 	Runtime two checks for the Cons (:) constructor are done.
% \end{frame}

\subsection{Type Systems}
\frame{
	\frametitle{Type inferencing}

	\begin{quotation}
    "Type inference refers to the ability to deduce automatically the type of an expression in a programming language."
    \end{quotation}
    
    \only<2->{For example the \emph{identity} function |\x->x : alpha -> alpha|}
}

\frame{
	\frametitle{Hindley-Milner}
    
	\begin{itemize}
	\item<1-> Damas-Milner
	\item<2-> Principle type
	\item<3-> Decidable inferencing
	\end{itemize}
}

\frame{
	\frametitle{Higher-rank types}
    
	\begin{itemize}
	\item<1-> Haskell '98 types are rank-1
	\item<2-> |a -> b -> a|
   \item<3-> |forall a b. a -> b -> a|
   \item<4-> |forall a. a -> (forall b. b -> a)|
   \item<5-> |forall b. (forall a. a -> a) -> b -> b|
	\end{itemize}

	\only<6->{Refers to the the number of |forall|s nested to the left of a |(->)|}
}

\frame{
	\frametitle{Higher-rank types}
    
	\begin{code}
	poly = \f -> (f 1, f 'c')
	\end{code}
	
	cannot be expressed without Higher-Rank types.
}

\frame{
	\frametitle{SystemF}
   Provides typing support for higher-rank functions

	\begin{block}<+->{Terms}
		\begin{itemize}
		\item Type abstraction ($\Lambda X. t$)
		\item Type application (t [T])
		\end{itemize}
	\end{block}

	\begin{block}<+->{Types}
		\begin{itemize}
		\item Type variables (X)
		\item Universal types (|forall X. T|)
		\end{itemize}
	\end{block}
}

\frame{
	\frametitle{SystemF}
   Typing of the |id| function
	\begin{itemize}
	\item<1-> $id = \lambda x. x$
	\item<2-> $id = \Lambda X. \lambda x : X -> x : X$
	\item<3-> Typing |id 3|
	\item<4-> id [Int] = [X$\rightarrow$Int]($\lambda x:X. x$)
	\end{itemize}
}

\frame{
	\frametitle{SystemF}
	\begin{itemize}
	\item<1->|poly = \f -> (f 1, f 'c')| is now typeable
	\item<2->Undecidable
	\item<3->Requires annotation
	\item<4->|poly = \(f :: forall a -> Int) -> (f 1, f 'c')|
	\end{itemize}
}

\frame{
	\frametitle{The problem}
   
	Type systems are specified using typing rules
   \only<2->{
		\begin{prooftree}
			\AxiomC{$x : \sigma \in \Gamma$}
		\LeftLabel{Var:\quad}
			\UnaryInfC{$\Gamma\vdash x : \sigma$}
		\end{prooftree}
	}
   \only<2->{However..}
}

\frame{
	\frametitle{The problem}
   
	Disconnect between typing rules and implementation
   \begin{itemize}
	\item<1-> Nondeterministic
	\item<2-> Implicit assumptions
	\item<3-> Implementation does not resemble typing rules
	\item<4-> Complexity explodes with size of AST
	\item<5-> A lot of it due to language used
	\end{itemize}
}

\frame{
	\frametitle{Goal}
    
	Implement type system using attribute grammars
	\begin{itemize}
	\item<1-> Easier to understand
	\item<2-> Easier to prove correct
	\item<3-> Easier to document and scale
	\end{itemize}
}

\frame{
	\frametitle{Contributions}

	\begin{itemize}
	\item<1-> Implementation using attribute grammars
	\item<2-> Implementation \& specification for the HML type system for EH
	\end{itemize}
}

\section{Attribute Grammars}

\frame{
	\frametitle{Context Free Grammars}

	\begin{itemize}
	\item<1-> Can only describe syntax
	\item<2-> Cannot specify any context-sensitive conditions
   \item<2-> $a^nb^nc^n$
	\item<3-> A way to define semantics/meaning
	\end{itemize}
}

\frame{
	\frametitle{Context Free Grammars}

	\begin{figure}[H]
\begin{grammar}
      [(colon){$\rightarrow$}]
      [(semicolon)$|$]
      [(comma){}]
      [(period){\\}]
      [(quote){\begin{bf}}{\end{bf}}]
      [(nonterminal){$\langle$}{$\rangle$}]
<Expr>   \hspace{52pt} : <number>; <Expr>, <operator>, <Expr>.
<number> \hspace{42pt} : <digit>;<digit>,<number>.
<digit> \hspace{55pt} : "0";"1";"2";"3";"4";"5";"6";"7";"8";"9".
<relational operator>: "$-$";"$+$".
\end{grammar}
\caption{BNF definition for Expressions}
\label{grammar:bnf:expr}
\end{figure}
}

\frame{
	\frametitle{Parse Trees}

	\begin{figure}[H]
	\centering
	\begin{tikzpicture}[>=stealth']
	\node (r0) [draw, circle] {+};
	\node (r1) [below left=1cm of r0, draw, circle] {2}
	  edge[<-] (r0.south);
	\node (r2) [below right=1cm of r0, draw, circle] {+}
	  edge[<-] (r0.south);
	\node (s0) [below left=1.4cm of r2, draw, circle] {-}
	  edge[<-] (r2.south);
	\node (s1) [below left=1cm of s0, draw, circle] {3}
	  edge[<-] (s0.south);
	\node (s2) [below right=1cm of s0, draw, circle] {5}
	  edge[<-] (s0.south);
	\node (f0) [below right=1.5cm of r2, draw, circle] {-}
	  edge[<-] (r2.south);
	\node (f1) [below left=1cm of f0, draw, circle] {6}
	  edge[<-] (f0.south);
	\node (f2) [below right=1cm of f0, draw, circle] {1}
	  edge[<-] (f0.south);
	\end{tikzpicture}
	\caption{Parse tree example for "2 + (3 - 5) + (6 - 1)"}
	\label{fig.example1.parsetree}
	\end{figure}

	\begin{itemize}
	\item<1-> AG's are additions to CFG
	\item<2-> Nodes are a production or a non-terminal
	\end{itemize}
}

\frame{
	\frametitle{}

}

\subsection{Introduction}

\subsection{UUAGC}

\subsection{Ruler-Core}

\section{HML}

\subsection{Introduction}

\subsection{Types}

\subsection{Invariants}

\subsection{Semantics}

\subsection{Utility functions}

\subsection{Normal Form}

\section{Implementation}

\subsection{Inferencing}

\subsection{Unification}

%\section{Additions}
%
%\subsection{Tuples}
%
%\subsection{Case expressions}

\section{Result}

\section{Problems and Future work}

\subsection{HML}

\subsection{Ruler-Core}

\section{Conclusion}
